\documentclass{article}
\usepackage{graphicx}
\usepackage{hyperref}
\usepackage{geometry}
\geometry{a4paper, margin=1in}

\title{Workshop 4: Simulation Report}
\author{Antigravity Agent}
\date{\today}

\begin{document}

\maketitle

\section{Introduction}
This report documents the implementation and results of the Workshop 4 Simulation project. The goal was to develop a reproducible software package that exercises a refined system architecture through two scenarios: a Data-driven ML Simulation and an Event-based Cellular Automata Simulation.

\section{System Architecture}
The system follows an eight-layer architecture as defined in Workshop 3.
\begin{enumerate}
    \item \textbf{Data Ingestion}: \texttt{src/ingestion.py} loads and validates data.
    \item \textbf{Processing}: \texttt{src/preprocessing.py} handles imputation and scaling.
    \item \textbf{Feature Store}: \texttt{src/features.py} manages derived features.
    \item \textbf{Modeling}: \texttt{src/models.py} trains Random Forest and MLP models.
    \item \textbf{Ensemble/Orchestration}: \texttt{src/experiments.py} manages the workflow.
    \item \textbf{Evaluation}: Metrics (RMSE, MAE) are calculated and logged.
    \item \textbf{Deployment/API}: Simulated via CLI scripts.
    \item \textbf{Feedback}: Drift detection triggers retraining.
\end{enumerate}

\section{Scenario 1: Data-driven ML Simulation}
We implemented a pipeline using \texttt{census\_starter.csv}.
\subsection{Methods}
\begin{itemize}
    \item \textbf{Preprocessing}: Median imputation, Standard Scaling.
    \item \textbf{Models}: Random Forest (n\_estimators=100) and MLP (64, 32 hidden units).
    \item \textbf{Drift Detection}: Kolmogorov-Smirnov test on feature distributions.
\end{itemize}

\subsection{Results}
The Random Forest model achieved an RMSE of approximately 0.07 (on scaled data).
\begin{figure}[h]
    \centering
    \includegraphics[width=0.8\textwidth]{figs/residuals.png} % Placeholder path, ensure image exists
    \caption{Actual vs Predicted Residuals}
    \label{fig:residuals}
\end{figure}

\section{Scenario 2: Cellular Automata Simulation}
We implemented a 2D CA model to simulate microenterprise density.
\subsection{Rules}
\begin{itemize}
    \item \textbf{Growth}: Probability increases with neighbor density.
    \item \textbf{Decay}: Random stochastic decay.
    \item \textbf{Perturbation}: Gaussian noise added at each step.
\end{itemize}

\subsection{Emergent Behavior}
The simulation shows clustering behavior over time, demonstrating how local rules lead to global structure.
\begin{figure}[h]
    \centering
    \includegraphics[width=0.8\textwidth]{figs/ca_final_state.png} % Placeholder path
    \caption{Final State of CA Grid}
    \label{fig:ca_state}
\end{figure}

\section{Chaos and Sensitivity}
Small perturbations (sigma=0.05) in the CA model lead to divergent grid states over long time horizons, illustrating sensitivity to initial conditions and noise.

\section{Conclusion}
The simulation successfully demonstrates the integration of ML and complex systems modeling within the proposed architecture.

\end{document}
